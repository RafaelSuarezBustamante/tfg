\chapter{Objetivos}
\label{chap:objetivos}

\noindent
Para este capítulo, la normativa indica:

«Concretar y exponer el problema a resolver describiendo el entorno de trabajo,
la situación y detalladamente qué se pretende obtener. Limitaciones y
condicionantes a considerar para la resolución del problema (lenguaje de
construcción, equipo físico, equipo lógico de base y de apoyo, etc.). Si se
considera necesario, esta sección puede titularse ``Objetivos e hipótesis de
trabajo''. En este caso, se añadirán las hipótesis de trabajo que el alumno, con
su TFG, pretende demostrar».

\section{Objetivo general}

El hito final que se pretende lograr, destacando el problema específico que
resuelve o la funcionalidad que aporta la aplicación o sistema desarrollado.


\section{Objetivos específicos}

Los objetivos específicos son las partes independientes del proyecto que tienen
valor por si mismas.

Por ejemplo, si el objetivo general fuera destruir una flota enemiga, los
objetivos específicos podrían ser: hundir el portaaviones, inutilizar las
torretas de los destructores, eliminar los cazas enemigos, etc.

Los objetivos específicos no son tareas; análisis, diseño, etc. no tienen valor
intrínseco para el cliente, si por ejemplo el proyecto se cancela en la fase de
diseño no se le entrega nada de valor al cliente, luego no se cubre ningún
objetivo.

No se deben confundir los objetivos del proyecto con los objetivos del
alumno. Indicar como objetivo que el alumno va a aprender o a estudiar
determinada disciplina o herramienta no aporta nada al cliente. Deben ser
entregables que el cliente puede valorar y por los que estaría dispuesto
a pagar. Resumiendo, son \textbf{objetivos}, no subjetivos.

\subsection{Objetivo 1}

\subsection{Objetivo 2}

\subsection{Objetivo 3}


% Local Variables:
%  coding: utf-8
%  mode: latex
%  mode: flyspell
%  ispell-local-dictionary: "castellano8"
% End:
