\chapter{Método de trabajo}
\label{chap:metodo}
\drop{T}ras haber analizado cuales son los objetivos a desarrollar en el TFG, y realizado un análisis de las tecnologías existentes, es necesario asignar una metodología para la gestión del proyecto. 

Para ello, y valorando el número de personas que están involucradas en el proyecto (director del TFG y su autor), se realiza una investigación sobre cuales son las alternativas de metodologías para llevarlo a cabo, que sea ágil, ligera, y por supuesto, flexible a cambios para adaptar nuevos requisitos que puedan surgir.

Las metodologías Agile, surgen en el año 1990, como herramienta de gestión software que indicase unas correctas directivas para llevar a acabo los proyectos. 

Las metodologías existentes hasta la época, tenían una rigurosa asignación de roles, actividades y artefactos (incluyendo el modelado y una documentación muy detallada), que no obstante, a día de hoy sigue siendo necesaria para proyectos de gran envergadura, que necesitan una alta gestión de tiempo y recursos. 

Según avanza el desarrollo de aplicaciones software, entran en escena nuevas situaciones en las cueles este tipo de métodos de gestión no encaja totalmente con estas metodologías tradicionales. 

Los métodos Agile promueven una gestión de proyectos, que se basa principalmente en fomentar la constante inspección del trabajo y la adaptación de éste. Se trata de un sistema organizado que facilita el trabajo el equipo, una correcta organización, y favorece el rendimiento del tiempo empleado en el desarrollo.

Scrum fue creado, con las características de estos métodos de gestión, trabajando con una comunicación directa y empleando ingeniería concurrente, basándose en las ideas del «Manifiesto por el Desarrollo Ágil de Software» (detalladas en la siguiente sección del documento).

Por lo tanto, basándose en estas ideas, y teniendo en cuenta el número de personas que están involucradas en el proyecto, se ha elegido la metodología de trabajo Scrum.


\section{Metodología Agile}



\section{Scrum como metodología de trabajo}



\subsection{Roles}



\subsection{Artefactos}



\subsection{Motivos de la elección del método Scrum}



\section{Aplicación del método de trabajo}
