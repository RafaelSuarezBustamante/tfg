\chapter{Método de trabajo}
\label{chap:metodo}
\drop{T}ras haber analizado cuales son los objetivos a desarrollar en el TFG, y realizado un análisis de las tecnologías existentes, es necesario asignar una metodología para la gestión del proyecto. 

Para ello, y valorado el número de personas que están involucradas en el proyecto (director del TFG y su autor), se realiza una investigación sobre cuales son las alternativas de metodologías para llevarlo a cabo, que sea ágil, ligera, y por supuesto, flexible a cambios para adaptar nuevos requisitos que puedan surgir.

Las metodologías Agile, surgen en el año 1990, como herramienta de gestión software que indicase unas correctas directivas para llevar a acabo los proyectos. 

Las metodologías existentes hasta la época, tenían una rigurosa asignación de roles, actividades y artefactos (incluyendo el modelado y una documentación muy detallada), que, no obstante, a día de hoy sigue siendo necesaria para proyectos de gran envergadura, que necesitan una alta gestión de tiempo y recursos. 

Según avanza el desarrollo de aplicaciones software, entran en escena nuevas situaciones en las cueles este tipo de métodos de gestión no encaja totalmente con estas metodologías tradicionales. 

Los métodos Agile promueven una gestión de proyectos, que se basa principalmente en fomentar la constante inspección del trabajo y la adaptación de éste. Se trata de un sistema organizado que facilita el trabajo el equipo, una correcta organización, y favorece el rendimiento del tiempo empleado en el desarrollo.

Scrum fue creado, con las características de estos métodos de gestión, mediante una comunicación directa y haciendo uso de ingeniería concurrente, basándose en las ideas del «Manifiesto por el Desarrollo Ágil de Software» (detalladas en la siguiente sección del documento).

Por lo tanto, basándose en estas ideas, y teniendo en cuenta el número de personas que están involucradas en el proyecto, se ha elegido la metodología de trabajo Scrum.


\section{Metodología Agile}



\section{Scrum como metodología de trabajo}



\subsection{Roles}



\subsection{Artefactos}



\subsection{Motivos de la elección del método Scrum}



\section{Aplicación del método de trabajo}

\subsection{Iteración 0: Inicio del TFG}
El autor de este Trabajo Fin de Grado, se reunió con distintos profesores de la Escuela de Ingeniería industrial de Toledo para contrastar las propuestas de las diferentes líneas de TFG ofertadas.

Tras analizar cada una de las líneas de trabajo, finalmente el autor de este TFG acuerda con \textbf{Francisco Moya Fernandez} realizar una primera reunión para analizar los puntos a tratar en la \textit{Línea de electronica de consumo y lúdico educativa}, ya que ofrecía la posibilidad de integrar y desarrollar los contenidos, capacidades y habilidades adquiridas durante todo el periodo de docencia en el \textbf{Grado en Ingeniería Industrial y Automática.}

Los puntos expuestos por \textbf{Francisco Moya Fernandez} sobre la línea de trabajo antes mencionada en esta 

La línea de trabajo antes mencionada, según expone \textbf{Francisco Moya Fernandez}, abarca el diseño y prototipado de cualquier sistema electrónico con aplicaciones lúdico-educativas, artísticas, o de uso doméstico. Típicamente se trata de sistemas empotrados basados en microcontroladores o microprocesadores (Arduino, Raspberry Pi, BeagleBone, Carambola, CHIP, Teensy, etc.) junto a un conjunto de periféricos y electrónica discreta específica de la aplicación. En las aplicaciones educativas, más que construir y diseñar juguetes, se intenta explotar la tecnología para facilitar los mecanismos de aprendizaje.

En esta primera iteación se decide realizar el diseño y desarrollo de una plataforma interactiva de juego de bajo coste, para el ámbito educativo, basada en interfaces tangibles y diseñada para niños de corta edad, ofreciendo un entorno gráfico simple que permita al niño ser capaz de diseñar secuencias y modificarlas interactuando con elementos tangibles.\

\subsubsection{Desarrollo del trabajo escrito y método de trabajo}
Se acuerda para el desarrollo de la memoria utilizar la herramienta de procesamiento de textos \LaTeX{} por la facilidad para redactar documentos estructurados.



\subsubsection{Próximas iteraciones e historias de usuario.}
Se acuerda establecer la \textbf{Metodología Scrum} adaptada, que fue expuesta en el capítulo \textbf{Método de trabajo}.\
Para la primera iteración se definen los siguientes pasos a seguir:
\begin{itemize}
\item Estudio del problema. Entender que hay que construir.
\item Primer análisis del problema.
\item Historia de usuario: preparar una primera propuesta de diseño según el primer análisis del problema.
\end{itemize}

\subsection{Iteración 1 : Formación y Planificación del TFG }

Esta primera iteración tiene como objetivo determinar que tipo de plataforma de juego se adapta a la programación tangible para niños, entendiendo como funcionan estas plataformas.

Estos sistemas permiten a los niños, escribir un programa mediante el uso de objetos físicos sin hacer uso de un teclado, para posteriormente ver los resultados realizados por ellos mismos.

Realizar el análisis de una primera propuesta basada en un sistema tangible, y dedicado a la programación tangible para niños. Identificar dentro de la propuesta que elementos se identifican con la programación tangible.


\subsubsection{Próximas iteraciones e historias de usuario.}
Quedan establecidos para la próxima iteración los siguientes puntos:
\begin{itemize}
\item Estudio avanzado del problema. Revisión sistemática (recopilación de información).
\item Identificar áreas de mejora en los actuales sistemas tangibles.
\item Selección de las áreas de mejora para obtener los objetivos.
\end{itemize}


\subsection{Iteración 2: Estudio avanzado del problema. Identificación y selección de las áreas de mejora en sistemas tangibles.}

Esta iteración se establece realizar un estudio avanzado del problema para obtener las áreas de mejora que se encuentran en los actuales sistemas de programación tangible para niños.


\subsubsection{Estudio avanzado del problema.}
En el \textbf{Capítulo Antecedentes} de este documento, se muestran los resultados sobre los principales estudios experimentales aplicados a la interacción tangible, y como motiva a niños de tempranas edades en sus habilidades computacionales. En ese capítulo se indican algunos de los dispositivos más importantes y que diferentes enfoques aportan en la interacción tangible, y que han sido utilizados para el aprendizaje computacional, así como sus características y tecnología utilizada. El estudio se realiza para identificar las áreas de mejora en sistemas tangibles.

\subsubsection{Identificar áreas de mejora en sistemas tangibles.}
Los diferentes sistemas de interacción tangible tienen como propósito general que la
información sea comprensible y literalmente captable, haciendo uso de objetos físicos
que sean manipulables de una forma natural.\\
Otra de las características de la aplicación de los sistemas tangibles en el ámbito
educativo, es el entorno visual, que debe de motivar fuertemente al usuario, disponiendo
de un entorno de fácil uso e intuitivo, y que además fomente la capacidad de
razonamiento, creación e imaginación de los niños.\\
La evolución de la tecnología aplicada a los modelos actuales de sistemas tangibles, van
adaptándose para mejorar los diseños y conseguir de esta forma sistemas más fiables.

\subsubsection{Próximas iteraciones e historias de usuario.}
Quedan establecidos para la próxima iteración los siguientes puntos:
\begin{itemize}
\item Establecer requisitos iniciales.
\item Identificar y redactar objetivos.
\item Investigación sobre las tecnologías a utilizar.
\item Historia de usuario: preparar una segunda propuesta de diseño según las áreas de mejora identificadas.
\end{itemize}


\subsection{Iteración 3: Establecer Requisitos Iniciales e Investigación sobre las Tecnologías a Utilizar}

Las áreas de mejora obtenidas en la Iteración 2 servirán de base para realizar la plataforma de juego. Se analiza cada punto para obtener la mejor solución posible, y que tecnología utilizar para cada caso.\

Establecer los requisitos a implementar en la plataforma de juego.

Exponer y analizar la segunda propuesta de diseño de juego, basada en los requisitos expuestos.


\subsubsection{Próximas iteraciones e historias de usuario.}
Quedan establecidos para la próxima iteración los siguientes puntos:
\begin{itemize}
\item Establecer objetivos específicos.
\item Finalizar estudio tecnológico para establecer tecnologías a utilizar.
\item Historia de usuario: sistema de localización entre ambos dispositivos.
\item Historia de usuario: procesamiento de datos del dispositivo de menor tamaño, y establecer tamaños en pantallas de ambos dispositivos
\end{itemize}

\subsection{Iteración 4: Establecer Objetivos y Tecnologías a Utilizar}

En esta iteración se define el objetivo general del presente Trabajo Fin de Grado, así como los objetivos específicos que definirán el transcurso del resto de iteraciones hasta obtener el resultado deseado. Los objetivos específicos podran ser modificados acorde con las necesidades del dispositivo.

\subsubsection{Objetivos del TFG}

\textbf{Objetivo general}: Diseñar y desarrollar una plataforma interactiva de juego de bajo coste, basada en interfaces tangibles, haciendo uso de las nuevas tecnologías, para promover el desarrollo de capacidades en niños de corta edad.

\textbf{Objetivos específicos}:
\begin{itemize}
\item Diseñar y programar un sistema de localización y reconocimiento de las interfaces de usuario tangibles.

\item Desarrollar un sistema eficiente de transferencia de datos entre ambas interfaces tangibles.

\item Realizar una correcta sincronización en la visualización de la aplicación entre ambas pantallas, que permita una buena representación de juego.

\item Ofrecer un entorno gráfico simple, que permita al niño ser capaz de diseñar secuencias y modificarlas interactuando con los elementos tangibles.

\item Adaptar sensores inerciales y ópticos para una mejor experiencia de juego, donde ambos elementos tangibles interactúen entre sí, permitiendo la manipulación de la información entre los dispositivos.

\item Diseñar un sistema de actualización de software de los dispositivos, a través de la conexión externa a un servidor.
\end{itemize}

\subsubsection{Sistema de localización entre ambos dispositivos}
De manera conjunta con las búsquedas de nuevas tecnologías, y antes de proceder a exponer que tecnologías utilizar, se formula la propuesta del sistema de localización entre dispositivos para ser valorada junto con \textbf{Francisco Moya Fernández}.
 
\subsubsection{Procesamiento de datos del dispositivo de menor tamaño, y establecer tamaños en pantallas de ambos dispositivos }

Para corregir los posibles problemas que puede causar en el entorno gráfico en el dispositivos se debe determinar el microcontrolador usado en los dispositivos.

Dado que la plataforma de juego está orientada a niños de corta edad, el tamaño de las pantallas debe ser el adecuado para una correcta visualización de los juegos.


\subsubsection{Tecnologías a utilizar.}

Identificadas las áreas de mejora que han permitido definir unos objetivos para la plataforma de juego, se exponen las tecnologias que van a ser implementadas.\

Los elementos que serán integrados, son elegidos acorde con las propuestas de diseño, de las cuales se extrajeron aquellos elementos que cumplían con los requisitos iniciales. Esta lista de componentes es una \textbf{lista abierta}, y podrá ser modificada durante el transcurso del sistema, de acuerdo a las necesidades del mismo:

\begin{itemize}
\item \textbf{Placa para el desarrollo.} Todos los eventos generados en la plataforma de juego para ambos dispositivos serán manejados mediante el computador de placa única \textbf{«Raspberry Pi 3»}. Esta placa dispone de un microprocesador «BCM2837» de 64 bits de cuádruple núcleo a 1,2GHz, 1GB de memoria RAM y GPU de doble núcleo Videocore IV, que permite manejar y procesar eventos e imágenes según los requisitos de la aplicación.
Integra el chip «BCM43143» para la comunicación Wi-Fi entre dispositivos, lo que elimina la necesidad de utilizar cables. Para el uso de la interfaz gráfica de usuario mediante dispositivo de salida (pantalla LCD), dispone de un puerto de vídeo/audio HDMI. Incorpora 40 pines GPIO, los cuales disponen de un bus \emph{I2C} \footnote{I2C, «Inter-Integrated Circuit».} para la utilización de sensores dentro del sistema de juego, y un bus «SPI»\footnote{SPI, «Serial Peripheral Interface» } como alternativa al puerto HDMI para mostrar la interfaz gráfica de usuario.
Su reducido tamaño, bajo consumo, además de contar con una gran comunidad de usuarios, hacen de este dispositivo el adecuado para la realización de un prototipo de plataforma de juego.
\item \textbf{Pantallas.} Para el dispositivo \emph{TUIO1}, es utilizada una pantalla LCD con un tamaño de 7 pulgadas. Este tamaño es adecuado para una correcta visualización de los contenidos del juego. La pantalla dispone de un panel multitáctil tipo capacitivo, que permitirá gestionar los eventos generados sobre el mismo. La conexión con la placa para el desarrollo para la visualización de la interfaz gráfica, se realiza mediante puerto HDMI.
Los eventos táctiles son administrados mediante conexión USB.\
El dispositivo \emph{TUIO2} dispone de una pantalla LCD, de un tamaño de 3,5 pulgadas que utiliza un panel táctil resistivo para captar los eventos de entrada. Este tamaño permite visualizar los contenidos en el desarrollo del juego, de manera que ofrece una interacción con la interfaz gráfica adecuada. La conexión con la placa para el desarrollo es mediante comunicación «SPI».
\item \textbf{Sensores.} Los sensores integrados en la plataforma de juego son, un sensor inercial \emph{MPU9250}, y un sensor de color \emph{TCS34725}. Ambos sensores dedicados a efectuar una interacción entre ambos dispositivos. El elemento \emph{TUIO2} dispone de ambos sensores, y es el encargado de recibir los eventos generados, para posteriormente transmitir dichos eventos vía Wi-Fi con el dispositivo \emph{TUIO1}.\

\item \textbf{Sistema operativo y entorno de programación.} Para el desarrollo de la aplicación, se utilizará la distribución gratuita \textbf{«KivyPie»}. Esta distribución incluye el sistema operativo \textbf{«PipaOS»} basado en \textbf{«Raspbian»}, utilizado en proyectos hardware desde cero, que contiene el software mínimo para poner «Raspberry Pi» en funcionamiento. Incluye la herramienta para desarrollo \textbf{«Kivy».} Esta herramienta es un entorno de trabajo para aplicaciones multitáctiles de \textbf{«Python».}
\end{itemize}

Las características expuestas, y la descripción de dispositivos y entornos de programación, están reflejadas en el \textbf{Capótulo 5. Tecnologías para el desarrollo.}.

\subsubsection{Próximas iteraciones e historias de usuario.}
Quedan establecidos para la próxima iteración los siguientes puntos:
\begin{itemize}
\item Iniciar sistema de localización en pantalla táctil capacitiva.
\item Programa de pruebas para las comunicaciones entre dispositivos. 
\end{itemize}

\subsection{Iteración 5: Inicio del sistema de localización entre dispositivos y pruebas en las comunicaciones }

En esta iteración se tratan dos de los puntos más importantes de la plataforma de juego, como es \textbf{la localización del dispositivo \emph{TUIO2}, sobre \emph{TUIO1}}, y como ambos elementos intercambian información mediante comunicación inalámbrica.

\subsubsection{Sistema de localización.}

El sistema de localización consiste en obtener las coordenadas de la posición del «Widget tangible TUIO2» sobre la pantalla táctil capacitiva del dispositivo \emph{TUIO1}, con el objetivo de generar un «efecto cristal», es decir, obtener una representación gráfica en la pantalla de \emph{TUIO2}, del cuadrante correspondiente a la representación gráfica de \emph{TUIO1}, y que cubre el dispositivo \emph{TUIO2}, como puede ser una imagen o una animación.\

\subsubsection{Pruebas de comunicación.}
Establecer la aquitectura para la comunicación entre dispositivos, así como el protocolo de comunicaciones utilizado.

\subsubsection{Próximas iteraciones e historias de usuario.}
Quedan establecidos para la próxima iteración los siguientes puntos:
\begin{itemize}
\item Leer datos unidad de medición inercial \emph{MPU9250}.
\item Diseño de una máquina de estados para las comunicaciones.
\item Comunicar datos unidad de medición inercial \emph{MPU9250}. Diseño de máquina de estados.
\end{itemize}

\subsection{Iteración 6: Lectura de datos y diseño de máquina de estados de la unidad de medición inercial \emph{MPU9250}. Diseño de máquina de estados para las comunicaciones.}
Uno de los objetivos específicos es adaptar sensores inerciales y ópticos para una mejor experiencia de juego, donde los elementos \emph{TUIO1} y \emph{TUIO2} interactúen entre sí, permitiendo la manipulación de la información entre los dispositivos.
La unidad de medición inercial integrada en la plataforma, es el sensor \emph{MPU9250}. Este sensor formará parte del dispositivo \emph{TUIO2}. Esta destinado principalmente a capturar eventos de movimiento para generar acciones durante el transcurso del juego. Estos eventos son filtrados y comunicados con el dispositivo \emph{TUIO1}.

\subsubsection{Lectura de datos de la unidad de medición inercial. Registros.}
El sensor \emph{MPU9250} es la combinación de dos sensores, \emph{MPU-6500} y \emph{AK8963}. El sensor \emph{MPU-6500}, que contiene un giroscopio de 3 ejes, un acelerómetro de 3 ejes y un procesador digital de movimiento \emph{DMP}\footnote{Data Management Platform}. \emph{AK8963} es un magnetómetro digital de tres ejes.
La comunicación con todo los registros del dispositivo se realiza utilizando \emph{I2C} a 400kHz.

\subsubsection{Diseño de máquina de estados de la unidad de medición inercial \emph{MPU9250}}
Para comunicar la lectura de registros de los datos obtenidos por el sensor \emph{MPU9250} se realiza un diseño de máquina de estados.

\subsubsection{Diseño de máquina de estados para las comunicaciones.}
Para comunicar de forma fluida los eventos generados en ambos dispositivos se diseña a partir del programa de pruebas expuesto en la \textbf{Iteración 5}, una máquina de estados \textbf{FSM}\footnote{finite state machine} para administrar las comunicaciones. Tanto el \textbf{lado cliente} como el \textbf{lado servidor}, tendrán la misma arquitectura de diseño. 

\subsubsection{Próximas iteraciones e historias de usuario.}
Quedan establecidas para la próxima iteración los siguientes puntos:
\begin{itemize}
\item Diseño software para la interfaz gráfica de usuario del dispositivo \emph{TUIO1}.
\item Diseño software para la interfaz gráfica de usuario del dispositivo \emph{TUIO2}.
\item Diseño de máquina de estados FSM para los dispositivos \emph{TUIO1} y \emph{TUIO2}
\item Lectura de datos y diseño de la máquina de estados para el sensor de color \emph{TCS34725}.
\end{itemize}

\subsection{Iteración 7: Diseño software de las interfaces gráficas de usuario}
Uno de los objetivos del presente TFG, es ofrecer un entorno gráfico simple, que permita al niño ser capaz de diseñar secuencias y modificarlas interactuando con los elementos tangibles.

\subsubsection{Diseño de máquina de estados FSM para los dispositivos \emph{TUIO1} y \emph{TUIO2}}
Para administrar los eventos generados sobre la platadorma de juego se implementa una máquina de estados en ambos dispositivos. 

\subsubsection{Lectura de datos del sensor de color \emph{TCS34725}}
El dispositivo \emph{TCS34725} proporciona un retorno digital de los valores de detección de luz roja, verde y azul (RGB). Dispone de un filtro de bloqueo \textbf{IR}\footnote{Infrared radiation} para una mejor detección de la luz ambiente.\
La integración de este dispositivo hace posible una interacción directa con el medio que le rodea, lo que permite ampliar las opciones de juego dentro de la plataforma.\
La comunicación con todo los registros del dispositivo se realiza utilizando \emph{I2C} a 400kHz.

\subsubsection{Diseño de máquina de estados del sensor de color \emph{TCS34725}}
Para comunicar la lectura de registros de los datos obtenidos por el sensor \emph{TCS34725} se realiza un diseño de máquina de estados.

\subsubsection{Próximas iteraciones e historias de usuario.}
Quedan establecidos para la próxima iteración los siguientes puntos:
\begin{itemize}
\item Diseño software de juego entre los dispositivos \emph{TUIO1} y \emph{TUIO2}. Manejo y comunicación de eventos generados por los sensores \emph{MPU9250}, \emph{TCS34725} y eventos táctiles.
\end{itemize}

\subsection{Iteración 8: Diseño software de juego en la plataforma.}
\subsubsection{Próximas iteraciones e historias de usuario.}
Se acuerda para la próxima iteración los siguientes puntos:
\begin{itemize}
\item Diseño de actualización software de la plataforma de juego.
\end{itemize}
\subsection{Iteración 9 Sistema de actualización software.}
