\chapter{Resumen}
Este Trabajo Fin de Grado consiste en el diseño y desarrollo de una plataforma
interactiva de juego de bajo coste, para el ámbito educativo, basada en
interfaces tangibles y diseñada para niños de corta edad. Ofrece un entorno
gráfico simple, para que el niño pueda diseñar secuencias y modificarlas
interactuando con los elementos tangibles.

En esta aplicación educativa, más que construir y diseñar un juguete, se intenta
explotar la tecnología para facilitar los mecanismos de aprendizaje.

Actualmente se ha demostrado que los sistemas tangibles ofrecen una serie de
beneficios en la educación y la creatividad. La finalidad es fomentar este
sistema en aplicaciones educativas, combinando para este propósito las mejoras
de los medios digitales utilizando entornos visuales, animaciones y sonidos.

Se han analizado los distintos enfoques y la tecnología aplicada en los actuales
sistemas de programación para niños mediante interacción tangible, mostrando los
resultados obtenidos en estudios experimentales aplicados a la interacción
tangible, y cómo mejoran las habilidades computacionales de los niños,
localizando las áreas de mejora de estos dispositivos, las cuales seran la
referencia y justificación del desarrollo de la aplicación que se presenta en
este TFG.

%% Esta frase no se entiende, no entiendo lo que quieres decir 



\chapter{Abstract}

English version of the previous page.
