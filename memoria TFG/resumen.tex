\chapter{Resumen}
Este Trabajo Fin de Grado consiste en el diseño y desarrollo de una plataforma interactiva de juego de bajo coste, basado en interfaces tangibles para aplicaciones lúdico-educativas, empleando sistemas empotrados basados en microprocesadores, junto a un conjunto de periféricos, aprovechando la tecnología para facilitar los mecanismos 
de aprendizaje. 

En esta aplicación educativa, más que construir y diseñar un juguete, se intenta explotar la tecnología para facilitar los mecanismos de aprendizaje.

Actualmente se ha demostrado que los sistemas tangibles ofrecen una serie de beneficios en la educación y la creatividad. La finalidad es fomentar este sistema en aplicaciones educativas, combinando para este propósito las mejoras de los medios digitales utilizando entornos visuales, animaciones y sonidos.

Se han analizado los distintos enfoques y la tecnología aplicada en los actuales sistemas de programación para niños mediante interacción tangible, mostrando los resultados obtenidos en estudios experimentales aplicados a la interacción
tangible, y cómo mejoran las habilidades computacionales de los niños.

El sistema de programación tangible diseñado en este proyecto, es un sistema electrónico con un entorno gráfico simple, que permite al niño ser capaz de diseñar secuencias y modificarlas interactuando con elementos tangibles.

Está formado por dos elementos principales. Ambos dispositivos
disponen de una pantalla táctil donde se desarrolla el juego. El intercambio de información se realiza utilizando comunicaciones inalámbricas \emph{WiFi}, por lo que son totalmente
independientes un dispositivo del otro.

El empleo de un sensor de color y una unidad de medición inercial, son herramientas que ofrecen la posibilidad al niño, de interactuar con el medio que les rodea, mejorando la experiencia de juego.

La idea principal parte de que los niños no sean simplemente consumidores de contenido y que tengan la posibilidad de crearlo, que ellos mismos sean capaces de tomar decisiones durante el juego de una manera tangible e intuitiva.


\chapter{Abstract}

English version of the previous page.
