\chapter{Objetivos}
\label{chap:objetivos}

\noindent
\drop{L}a propuesta de este TFG se plantea para realizar una plataforma de juego educativa, basada en interacción tangible, aplicando las nuevas tecnologías para fomentar la capacidad de los niños, a la hora de utilizar su razonamiento en el proceso creativo en el transcurso de creación de un juego. La idea principal parte de que los niños no sean simplemente consumidores de contenido y que tengan la posibilidad de crearlo, que ellos mismos sean capaces de tomar decisiones durante el juego de una manera tangible e intuitiva. 


\section{Objetivo general}

Diseñar y desarrollar una plataforma interactiva de juego de bajo coste, basada en interfaces tangibles y haciendo uso de las nuevas tecnologías, para promover el desarrollo de capacidades en niños de corta edad.

\section{Objetivos específicos}

\begin{itemize}
\item Diseñar y programar un sistema de localizacion y reconocimiento de las interfaces de usuario tangibles.

\item Desarrollar un sistema eficiente de transferencia de datos entre ambas interfaces tangibles.

\item Realizar una correcta sincronización en la visualización de la aplicación entre ambas pantallas, que permita una buena representación de juego.

\item Ofrecer un entorno gráfico simple, que permita al niño ser capaz de diseñar secuencias y modificarlas interactuando con los elementos tangibles.

\item Adaptar sensores inerciales y ópticos para una mejor experiencia de juego, donde ambos elementos tangibles interactúen entre sí, permitiendo la manipulación de la información entre los dispositivos.

\item Diseñar un sistema de actualización de software de los dispositivos, a través de la conexión externa a un servidor.
\end{itemize}



% Local Variables:
%  coding: utf-8
%  mode: latex
%  mode: flyspell
%  ispell-local-dictionary: "castellano8"
% End:
