\chapter{Interacción Persona-Ordenador e Interfaces}
\label{chap:interfaces}
\drop{A} lo largo de la historia la interfaz ha ido evolucionando ofreciendo un amplio abanico de opciones,
especializándose y diversificándose para dar lugar a nuevos modelos de interacción, como son las basadas en soportes móviles, interfaces gráficas, tangibles, de realidad aumentada, las interfaces cooperativas y colaborativas, etc. Todas ellas poseen una serie de ventajas y restricciones a la hora de adaptarse a distintos entornos y situaciones.


\section{Interación Persona-Ordenador}
La Interacción Persona-Ordenador (IPO o HCI: Human Computer Interaction), es la disciplina enfocada al estudio de la interacción realizada entre sistemas computacionales y las personas. El objetivo principal es el de mejorar esta interacción, consiguiendo que los sistemas computacionales sean más manejables, de manera que la interacción sea lo más intuitiva posible. 

La ACM\footnote{Association for Computing Machinery.} define IPO como: \emph{La disciplina encargada del diseño, evaluación e implementación de sistemas computacionales interactivos para uso humano y del estudio de los que rodea.}

El área de investigación de IPO, es aplicada en el mundo empresarial y académico, siendo una de las razones de cambio en el uso de la informática, mejorando los sistemas a medida que se localizaban nuevas necesidades de usuarios. Un ejemplo de ello son las interfaces gráficas de usuario, que revolucionaron y aumentaron las posibilidades de utilización en el uso de ordenadores. A día de hoy para el desarrollo de interfaces software, es posible hacer uso de bibliotecas y paquetes ya existentes, que facilitan la programación de los desarrolladores. El crecimiento de World Wide Web (WWW) es el resultado directo de investigaciones IP0.

La evolución de esta disciplina es debido principalmente a factores como:
\begin{itemize}
\item Creatividad humana: factor fundamental que determina qué necesidad se desea cubrir, y como sería del diseño de la misma. Ya en los primeros pasos dados en la ciencia de la informática, visionarios realizaron proyecciones imaginarias de cómo serían los ordenadores, y que funciones podrían ser capaces de llevar acabo.
\item El estado del arte en la tecnología: a menudo actuá como límite al diseño.
\item El mercado que engloba los ordenadores: está directamente en relación con el precio de los elementos, y que cantidad de usuarios utilizarán dicho producto, además del uso que harán de él.
\end{itemize}

Según Shackel \cite{Shackel}, a principios de los años 50, ordenadores tenían como objetivo la investigación, pensados principalmente para resolver problemas matemáticos y científicos, donde la fiabilidad de los cálculos era primordial. En la década de los 60 y 70 son fabricadas las primeras macrocomputadoras, dirigidas a profesionales en el tratamiento de datos. Estas máquinas tenían unos altos costes de fabricación y poca flexibilidad, además de las dificultades a la hora de ser programadas. En los años 70, las minicomputadoras aparecen en el mercado, enfocadas a ingenieros y otros profesionales, y donde aún la programación es elevada y compleja. Sobre los años 80 aparecen los ordenadores dirigidos a todo tipo de consumidores, donde la usabilidad toma especial importancia, ya que intenta satisfacer las necesidades y requerimientos que demandan los usuarios. Durante la siguiente década, surgen nuevos dispositivos de menor tamaño, donde la usabilidad sigue siendo su principal característica, presentando nuevas dificultades a los profesionales e investigadores de la IPO.

IPO analiza conjuntamente los aspectos que dependen del ordenado, y los que son humanos. Dependiendo del punto de vista desde que se realiza el estudio, el enfoque es tratado desde los siguientes contextos:
\begin{itemize}
\item Contexto humano: es completado por ciencias tales como psicología, ciencias cognitivas, de comunicación, de diseño gráfico e industrial, entre otras.
\item Computadoras y maquinaria: comprende los lenguajes de programación, gráficos por ordenador, sistemas operativos, y desarrollo de ambientes.
\end{itemize}

En la Figura~\ref{fig:HCI}, se muestran las disciplinas que intervienen en IPO.

\begin{figure}[!h]
\begin{center}
\includegraphics[width=0.5\textwidth]{HCI.pdf}
\caption{Ciencias que se relacionan con el HCI. (Fuente: Interacción tangible en aplicaciones educativas \cite{Artola})}
\label{fig:HCI}
\end{center}
\end{figure}

La interacción entre personas y tecnología se realiza por medio de un componente implícito: \textbf{La Interfaz}.
La perfecta interfaz debe entender las necesidades de los usuarios, sus estilos de interacción natural, apoyándoles para eliminar los problemas en el uso de herramientas computacionales. Para el correcto desarrollo de una interfaz, es preciso comprender todos los factores implicados (organizativos, ergonómicos, psicológicos, y sociales), los cuales determinan la manera de actuar de los usuarios y como ellos hacen uso de los ordenadores, para conseguir desarrollar herramientas donde los sistemas diseñados satisfagan los requisitos del usuario, y así, conseguir una interacción eficiente, efectiva y segura.


\section{Interfaz Gráfica de Usuario GUI}
La interfaz gráfica de usuario \emph{GUI}\footnote{del inglés, Graphical User Interfaz} hace uso de un conjunto de imágenes y objetos de forma gráfica, para mostrar la información proporcionando un entorno visual sencillo, que facilita la interacción \emph{HCI}\footnote{Human-Computer Interaction}. Algunas de estas \emph{GUI}, son diseñadas para un uso específico, como son las pantallas táctiles, que simplifican aún más la interacción \emph{HCI} mediante el sentido del tacto.
La interfaz entre personas e información digital requiere de dos componentes fundamentales: la entrada/salida o control, y representación. Los controles permiten a los usuarios manipular la información, mientras que las representaciones externas son percibidas por medio de los sentidos. La Figura~\ref{fig:GUI}, muestra el modelo básico de una interacción con interfaz gráfica. En ella se hace uso de un elemento de entrada, como controlador remoto (ratón de ordenador), cuya información digital es procesada, para ser representada de manera intangible, usando un monitor, o produciendo un sonido determinado. Es decir, el usuario interactúa mediante un dispositivo a distancia y, en última instancia, experimenta una representación intangible de información digital (pixeles y sonido).

\begin{figure}[!h]
\begin{center}
\includegraphics[width=0.5\textwidth]{GUI.pdf}
\caption{Modelo de interacción de las interfaces gráficas (Adaptado de Ishii \cite{Ishii})}
\label{fig:GUI}
\end{center}
\end{figure}

\section{Interfaz Tangible de Usuario TUI}
Las interfaces de usuario tangibles \emph{TUI} (Tangible User Interface), en su área de estudio, tiene como objetivo analizar el paradigma de interacción \emph{HCI}, no limitándolo a una simple pantalla. Hacer que la información sea comprensible y literalmente captable, es la finalidad principal de la interacción tangible, al proporcionar representación física de los datos digitales mediante objetos que los representen \emph{TUIOs} (Tangible User Interface Objects). Estos objetos pueden ser utilizados con la manipulación natural de los usuarios, realizando un puente de unión entre el mundo digital y el mundo real.

El objetivo detrás de las interfaces tangibles, es permitir la interacción con las computadoras a través de objetos familiares, combinando la experiencia del usuario en el mundo táctil, con el poder de la tecnología \cite{Ishii}.
Tal interacción física es básicamente de tipo unidireccional, dirigida desde el usuario al sistema, limitando los posibles patrones de interacción. En otras palabras, el sistema no tiene medios para apoyar activamente la interacción física.
La Figura ~\ref{fig:TUI} ilustra la idea clave de hacer uso de la representación tangible (física y captable) como control.

Esta representación tangible ayuda a superar la barrera entre lo físico y lo digital. En contra del modelo \emph{GUI} propuesto en la Figura ~\ref{fig:GUI}, que hace uso de un ratón como elemento de control, en este caso se hace uso de un prototipo tangible, que intenta incorporar la información digital en forma física. A través de la manipulación física de las representaciones tangibles, la información digital es alterada, mostrando dichos cambios, bien mediante salida en la representación tangible, como en la salida de la representación intangible.


\begin{figure}[!h]
\begin{center}
\includegraphics[width=0.5\textwidth]{TUI.pdf}
\caption{Modelo de interacción de las interfaces tangibles (Adaptado de Ishii \cite{Ishii})}
\label{fig:TUI}
\end{center}
\end{figure}


Aunque la representación tangible permite que la realización física se acople directamente a la información digital, tiene capacidad limitada para representar el cambio de muchas propiedades materiales o físicas. A diferencia de los píxeles en la pantalla de un ordenador, es muy difícil cambiar un objeto físico en su forma, posición o propiedades (por ejemplo, color, tamaño) en tiempo real. Para complementar esta limitación, \emph{TUI} también utiliza representaciones tales como proyecciones de video y sonidos, para acompañar las representaciones tangibles en el mismo espacio, dando expresión dinámica de la información digital subyacente.
El éxito de una \emph{TUI} a menudo se basa en un equilibrio y fuerte acoplamiento perceptual entre las representaciones tangibles e intangibles. Tanto las representaciones tangibles e intangibles se acoplan perceptualmente para lograr una interfaz transparente, que media activamente la interacción con la información digital y borre, de manera apropiada, el límite entre lo físico y lo digital. La coincidencia de entrada y salida junto con una respuesta en tiempo real, son requisitos importantes para lograr este objetivo.

\subsection{Uso de TUI en la educación}

Los sistemas tangibles se están convirtiendo en una alternativa a la interfaz gráfica de usuario. Diferentes marcos de diseño se han aplicado al desarrollo de estos sistemas, abriendo la puerta a nuevas formas de relacionarse con el aprendizaje \cite{Marshall}.\\

Un diseño tangible, ofrece libertad para poder explorar y manipular objetos físicos y observar que efectos producen sobre el mundo digital. La teoría del aprendizaje y la cognición justifica la asimilación de los conceptos teóricos donde se incluyen las prácticas de participación, construcción de modelos, y la actividad colaborativa entre otros. El aprendizaje realizado por medio de elementos tangibles tiene el potencial de que los niños pueden combinar lo conocido y familiar, en nuevas formas \cite{Manches}.\\

En el área del cálculo y sus conceptos abstractos (en los primeros años de educación de los niños), existen varios argumentos acerca de los beneficios de la manipulación de elementos tangibles. Esta manipulación de elementos físicos donde interviene la memoria visual podría guiar a la resolución de problemas. Este hecho pone de manifiesto, que los niños pequeños pueden llegan a reconocer ciertas cosas sin ser capaces de expresare a través del lenguaje, o sin ser capaces de reflexionar sobre lo que pueden llegar a conocer en un sentido explícito \cite{Malley}.


\section{Interfaces Sistema Multi-Pantalla}
Un sistema multi-pantalla o multi-monitor, consisten en combinar varios dispositivos de visualización con el fin de aumentar el espacio visual disponible para ejecutar una o varias tareas.
Inicialmente este tipo de interfaz fue diseñada con el propósito de mostrar la misma imagen en los distintos dispositivos conectados. Este tipo de uso era aplicado principalmente en las presentaciones, donde el usuario podría disponer de un duplicado de la imagen proyectada.

De manera posterior, los fabricantes aprovecharon esta tecnología para ampliar las aplicaciones posibles, incorporando visualizaciones independientes, aumentando el rango de visión de las aplicaciones, etc.

Son muy distintos los ámbitos de aplicación de esta tecnología, pudiéndose incorporar en los sistemas de control en la industria, sistemas de video-vigilancia, fabricación de juegos, etc.
El uso de varias pantallas en la industria de los videojuegos fue introducida por Nintendo en 1980 con su serie de consolas portátiles \emph{Game and Watch}.

